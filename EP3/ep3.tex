\documentclass[11pt]{extarticle}

\usepackage[portuguese]{babel}
\usepackage[latin1]{inputenc}
\usepackage{graphicx}
\usepackage{amsthm}
\usepackage{float}
\usepackage{caption}

\author{Augusto César - 8941234 \\Paulo Araújo - 8941112\\Eric Lee - 7557095}
\title{MAC0209 Modelagem e Simulação \\ EP3}

\begin{document}
\maketitle{}
\paragraph{}
Exercício 14.5 do livro "An Introduction to Computer Simulation Methods Applications to Physical System"
\begin{enumerate}
  \renewcommand{\theenumi}{\alph{enumi}}
\item                           % a
  O gráfico do diagrama fundamental pode ser usado para entender a relação do vluxo (velocidade) com a densidade (trânsito) em um sistema de tráfego.
  Pode-se notar que quanto menor o fluxo, o sistema tende a ter uma maior densidade, assim como uma velocidade muito baixa pode causar um não aproveitamento do espaço. Sendo assim, observando o diagrama fundamental podemos obter onde pode-se encontrar um ponto ótimo para que o sistema entre em equilíbrio.
  Os engarrafamentos ocorrem quando temos uma densidade maior que a densidade no ponto máximo do gráfico (ponto onde temos a máxima capacidade de carros). Após esse ponto, o fluxo e a velocidade passam a diminuir devido ao engarafamento.
\item                           % b
  Para obtermos resultados independentes do comprimento da estrada, temos que ter um comprimento mínimo do dobro de número de carros.
\item                           % c
  
\item                           % d
  $v_{max} = 1 : Flow \approx 0.09$ \\
  Muito espaço entre os carros, poucos engarrafamentos. Flow muito pequeno, demora muito pra um carro
  atravessar a rodovia.
  
  $v_{max} = 2 : Flow \approx 0.246$\\
  Pouco espaço entre os carros, durante o percurso surgem alguns engarrafamentos bem pequenos. A velocidade dos carros é quase constante durante toda a rodovia. O flow é bem grande se comparado a travessia com $v_{max} = 1$, demora pouco tempo pros carros atravessarem a rodovia.

  $v_{max} = 5 : Flow \approx 0.293$\\
  Espaco entre os carros variados, i. e., alguns carros muito próximos enquanto outros estão muito distantes. Surgem vários engarrafamentos grandes, enquanto a velocidade e o flow ficam variando muito.
\item                           % e
  $p = 0.2$ \\
  Velocidade dos carros quase constante, poucos engarrafamentos surgem durante o percurso. O flow é bem alto.

  $p = 0.8$ \\
  Velocidade dos carros varia muito, muitos engarrafamentos surgem durante o percurso. O flow é quase a metade de quando $p = 0.2$.
\item                           % f
  Quando adicionamos as rampas de entrada e saída na rodovia, podemos perceber um pequeno engarrafamento surgindo quando o carro que deseja sair naquela rampa está próximo de lá e começa a diminuir a sua velocidade. Esse engarrafamento acaba sendo proporcional a velocidade máxima dos carros, i. e., quando os carros estão muito rápidos o engarrafamento é bem grande, diferentemente de quando a velocidade máxima da rodovia é menor, onde surgem engarrafamentos pequenos e por menos tempo. \\
  Podemos perceber o efeito resultante de ter colocado essas rampas usando a probabilidade de um carro diminuir a sua velocidade igual a 0. Sendo assim, a velocidade dos carros e o flow da rodovia só poderão ser mudados pelo efeito das rampas. Diferente de quando não tinhamos as rampas, os carros não possuem mais um movimento constante (o que era percebido quando usavamos $p = 0$ sem as rampas). 
\item                           % g
  O comportamento em duas vias em relação à uma via tem impacto significativo no fluxo geral, pois, junto à mudança, é inserido um novo componente chamado ultrapassagem.
  A ultrapassagem permite que veículos de maior velocidade possam transpor as vias sem haver uma limitação estabelecida pelo veículo adiante, havendo condições (desejo de ultrapassar e espaço para ultrapassagem).
\item                           % h
  
\end{enumerate}

\end{document}
