\documentclass[11pt]{extarticle}

\usepackage[portuguese]{babel}
\usepackage[latin1]{inputenc}
\usepackage{graphicx}
\usepackage{amsthm}
\usepackage{float}
\usepackage{caption}

\author{Augusto C�sar - 8941234 \\Paulo Ara�jo - 8941112\\Eric Lee - 7557095}
\title{MAC0209 Modelagem e Simula��o \\ EP3}

\begin{document}
\maketitle{}
\paragraph{}
Exerc�cio 14.5 do livro "An Introduction to Computer Simulation Methods Applications to Physical System"
\begin{enumerate}
  \renewcommand{\theenumi}{\alph{enumi}}
\item                           % a
\item                           % b
\item                           % c
  
\item                           % d
  $v_{max} = 1 : Flow \approx 0.09$ \\
  Muito espa�o entre os carros, poucos engarrafamentos. Flow muito pequeno, demora muito pra um carro
  atravessar a rodovia.
  
  $v_{max} = 2 : Flow \approx 0.246$\\
  Pouco espa�o entre os carros, durante o percurso surgem alguns engarrafamentos bem pequenos. A velocidade dos carros � quase constante durante toda a rodovia. O flow � bem grande se comparado a travessia com $v_{max} = 1$, demora pouco tempo pros carros atravessarem a rodovia.

  $v_{max} = 5 : Flow \approx 0.293$\\
  Espaco entre os carros variados, i. e., alguns carros muito pr�ximos enquanto outros est�o muito distantes. Surgem v�rios engarrafamentos grandes, enquanto a velocidade e o flow ficam variando muito.
\item                           % e
\item                           % f
  
\item                           % g
\item                           % h
  
\end{enumerate}

\end{document}